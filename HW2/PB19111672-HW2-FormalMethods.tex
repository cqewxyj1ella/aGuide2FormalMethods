\documentclass[a4paper,10pt]{article}

\usepackage[table]{xcolor}
\usepackage[ruled]{algorithm2e}
\usepackage{graphicx}
\usepackage{fullpage}
\usepackage[fleqn]{amsmath} % [fleqn] enables align to flush to the left
\usepackage{boxedminipage}
\usepackage{amssymb,amsthm}
\usepackage[totalwidth=166mm,totalheight=240mm]{geometry}
\usepackage{tikz}
\usetikzlibrary{arrows}
\usepackage{CJK} % enables Chinese spelling
\input zhwinfonts
\usepackage{empheq}
\newcommand*\widefbox[1]{\fbox{\hspace{2em}#1\hspace{2em}}}

\newtheorem{theorem}{Theorem}[section]
\newtheorem{lemma}[theorem]{Lemma}
	
\parindent0mm
%\pagestyle{empty}

\newcommand{\nop}[1]{}

\newcounter{aufgc}
\newenvironment{exercise}[1]
{\refstepcounter{aufgc}\textbf{Exercise \arabic{aufgc}} \emph{#1}\\}
{

  \hrulefill\medskip}

\renewcommand{\labelenumi}{(\alph{enumi})}

\begin{document}
\begin{minipage}[b]{0.58\textwidth}
\large 
School of Computer Science and Technology\\
University of Science and Technology of China
\end{minipage}

\hrulefill
\vspace{0.2cm}
\begin{center}
  {\large \bf Exercise-2 for \\[1mm]
A Guide to Formal Methods\\[0.5mm]
2022 Spring}\\
  \bigskip

\end{center}
\vspace{0.1cm}

\begin{flushright}
	\begin{CJK}{UTF8}{song}
		PB19111672 徐怡 \\
	\end{CJK}
	Mar. 12th, 2022
\end{flushright}

From the  last page of the \textbf{Chap 3's PPT}. 

\hrulefill\medskip

% exercise 1
\begin{exercise}{}
Prove the validity of the following sequents:
\end{exercise}

\begin{enumerate}

	\item[(1)]	% Q1
		$(p\land q)\land r,\ s\land t\vdash q\land s$ \\
	\begin{align*}
		&1\ (p\land q)\land r 	&premise \\
		&2\ p\land q 			&\land e_1 1 \\
		&3\ q 					&\land e_2 2 \\
		&4\ s\land t 			&premise \\
		&5\ s 					&\land e_1 4 \\
		&6\ q\land s 			&\land i\ 3,5 \\
	\end{align*}

	\item[(2)]	% Q2
		$q\to r\vdash (p\to q)\to(p\to r)$ \\
	\begin{align*}
		&\quad\quad 1\ q\to r \quad\quad\quad\quad\quad\quad\quad\quad\quad premise \\
		&\fbox{
			$\begin{array}{l}
			\quad 2\ p\to q \quad\quad\quad\quad\quad\quad\quad\quad\quad assumption\\
	 			\fbox{
	 				$\begin{array}{l}
					3\ p \quad\quad\quad\quad\quad\quad\quad\quad\quad\quad\quad assumption\\
					4\ q \quad\quad\quad\quad\quad\quad\quad\quad\quad\quad\quad \to e\ 2,3 \\
					5\ r  \quad\quad\quad\quad\quad\quad\quad\quad\quad\quad\quad \to e\ 1,4 \\
		 			\end{array}$
				} \\
			\quad 6\ p\to r \quad\quad\quad\quad\quad\quad\quad\quad\quad \to i\ 3-5 \\
	 		\end{array}$
 		} \\
 		&\quad\quad 7\ (p\to q)\to(p\to r) \quad\quad\quad\to i\ 2-6 \\
	\end{align*}

	\item[(3)]	% Q3
		$\vdash q\to (p\to (p\to (q\to p)))$ \\
	\begin{align*}
		&\fbox{
			$\begin{array}{l}
			\quad 1\ q \ \quad\quad\quad\quad\quad\quad\quad\quad\quad\quad\quad assumption\\
	 			\fbox{
	 				$\begin{array}{l}
					2\ p \ \quad\quad\quad\quad\quad\quad\quad\quad\quad\quad\quad assumption\\
					3\ q\to p \ \quad\quad\quad\quad\quad\quad\quad\quad\quad \to i\ 1,2 \\
					4\ p\to (q\to p)  \ \quad\quad\quad\quad\quad\quad \to i\ 2,3 \\
		 			\end{array}$
				} \\
			\quad 5\ p\to (p\to (q\to p)) \ \ \quad\quad\quad \to i\ 2-4 \\
	 		\end{array}$
 		} \\
 		&\quad\quad 6\ q\to (p\to (p\to(p\to q))) \quad\to i\ 1-5 \\
	\end{align*}
	
	\newpage 	% a new page
	
	\item[(4)]	% Q4
		$p\to q\land r\vdash (p\to q)\land(p\to r)$ \\
	\begin{align*}
		&\fbox{
			$\begin{array}{l}
			\quad 1\ p \quad\quad\quad\quad\quad\quad\quad\quad assumption\\
			\quad 2\ p\to (q\land r) \ \quad\quad\quad \to premise \\
			\quad 3\ q\land r \ \quad\quad\quad\quad\quad\quad \to e\ 1,2 \\
			\quad 4\ q \quad\quad\quad\quad\quad\quad\quad\quad \land e_1 3 \\
			\quad 5\ r \quad\quad\quad\quad\quad\quad\quad\quad \land e_2 3 \\
	 		\end{array}$
 		} \\
 		&\quad\quad 6\ p\to q \quad\quad\quad\quad\quad\quad \to i\ 1-4 \\
 		&\quad\quad 7\ p\to r \quad\quad\quad\quad\quad\quad \to i\ 1-5 \\
 		&\quad\quad 8\ (p\to q)\land(p\to r)\quad \land i\ 6,7 \\
	\end{align*}
	
	\item[(5)]	% Q5
		$p\land \lnot p \vdash \lnot(r\to q)\land(r\to q)$ \\
	\begin{align*}
		&1\ p\land\lnot p 	&premise \\
		&2\ p 				&\land e_1 1 \\
		&3\ \lnot p 			&\land e_2 1 \\
		&4\ \perp 			&\lnot e\ 2,3 \\
		&5\ \lnot(r\to q)\land(r\to q)	&\perp e\ 4 \\
	\end{align*}
\end{enumerate}
\hrulefill\medskip

% exercise 2
\vspace{0.2cm}
\begin{exercise}{}
Prove the validity of the following sequents in predicate logic, where $P$ and $Q$ have arity 1, and $S$ has arity 0 (a 'propositional atom'):
\end{exercise}

\begin{enumerate}
	\item[(1)]	% Q1
		$\exists x(S\to Q(x))\vdash S\to\exists x Q(x)$ \\
	\begin{align*}
		&\quad\quad 1\ \quad\quad\quad\exists x(S\to Q(x)) \quad\quad\quad\quad premise \\
		&\fbox{
			$\begin{array}{l}
			\quad 2\ \quad x_0 \quad S\to Q(x_0) \quad\quad\quad\quad\quad assumption\\
	 			\fbox{
	 				$\begin{array}{l}
					3\ \quad\quad\quad S \quad\quad\quad\quad\quad\quad\quad\quad\quad assumption\\
					4\ \quad\quad\quad Q(x_0) \quad\quad\quad\quad\quad\quad\quad \to e\ 2,3 \\
					5\ \quad\quad\quad \exists x Q(x)  \ \ \quad\quad\quad\quad\quad\quad \exists x\ i\ 4 \\
		 			\end{array}$
				} \\
			\quad 6\ \quad\quad\quad S\to \exists x Q(x) \quad\quad\quad\quad \to i\ 3-5 \\
	 		\end{array}$
 		} \\
 		&\quad\quad 7\ \quad\quad\quad S\to\exists x Q(x) \quad\quad\quad\quad\quad \exists e\ 1,2-6 \\
	\end{align*}
	
	\item[(2)]	% Q2
		$\forall x P(x)\to S\vdash\exists x(P(x)\to S)$ \\
	\begin{align*}
		&1\ \forall x P(x)\to S 	&premise \\
		&2\ P(t)\to S 			&\forall x\ e\ 1,\ x\ is\ not\ free\ in\ S \\
		&3\ \exists x(P(x)\to S)	&\exists x\ i\ 2
	\end{align*}
	
	\newpage 	% a new page
	
	\item[(3)]	% Q3
		$\forall x(P(x)\land Q(x))\vdash\forall x P(x)\land\forall x Q(x)$ \\
	\begin{align*}
		&1\ \forall x(P(x)\land Q(x)) 	&premise \\
		&2\ P(t)\land Q(t) 				&\forall x\ e\ 1 \\
		&3\ P(t)						&\land e_1\ 2 \\
		&4\ Q(t)						&\land e_2\ 2 \\
		&5\ \forall x P(x)				&\forall x\ i\ 3 \\
		&6\ \forall x Q(x)				&\forall x\ i\ 4 \\
		&7\ \forall x P(x)\land\forall x Q(x)	&\land i\ 5,6 \\
	\end{align*}
	
	\item[(4)]	% Q4
		$\lnot\forall x\lnot P(x)\vdash\exists x P(x)$ \\
	\begin{align*}
		&\quad\quad 1\ \quad\quad\quad \lnot\forall x\lnot P(x) \quad\quad\quad premise \\
		&\quad\quad 2\ \quad\quad\quad \exists x\lnot\lnot P(x) \quad\quad\quad semantically\ understanding \\
		&\fbox{
			$\begin{array}{l}
			\quad 3\ \quad x_0 \quad \lnot\lnot P(x_0) \quad\quad\quad\quad assumption\\
			\quad 4\ \quad\quad\quad P(x_0) \quad\quad\quad\quad\quad \lnot\lnot e\ 3 \\
			\quad 5\ \quad\quad\quad \exists x P(x) \ \quad\quad\quad\quad \exists x\ i\ 4 \\
	 		\end{array}$
 		} \\
 		&\quad\quad 6\ \quad\quad\quad \exists x P(x) \ \quad\quad\quad\quad \exists e\ 2,3-5 \\
	\end{align*}
	
	\item[(5)]	% Q5
		$\forall x\lnot P(x)\vdash\lnot\exists x P(x)$ \\
	\begin{align*}
		&\quad\quad 1\ \quad\quad\quad \forall x\lnot P(x) \quad\quad\quad\quad premise \\
		&\fbox{
			$\begin{array}{l}
			\quad 2\ \quad\quad\quad \exists x P(x) \ \ \quad\quad\quad\quad assumption \\
	 			\fbox{
	 				$\begin{array}{l}
					3\ \quad x_0 \quad P(x_0) \quad\quad\quad\quad\quad assumption\\
					4\ \quad\quad\quad \lnot P(x_0) \ \quad\quad\quad\quad \forall x\ e\ 1 \\
					5\ \quad\quad\quad \perp \ \quad\quad\quad\quad\quad\quad \lnot e\ 3,4 \\
		 			\end{array}$
				} \\
			\quad 6\ \quad\quad\quad \perp \ \quad\quad\quad\quad\quad\quad \exists x\ e\ 2,3-5 \\
	 		\end{array}$
 		} \\
 		&\quad\quad 7\ \quad\quad\quad \lnot\exists x P(x) \quad\quad\quad\quad \lnot i\ 2-6(MT) \\
	\end{align*}
\end{enumerate}

\end{document}